The evaluation is done on thermal images
collected and labeled by thermographic camera, Optris PI450, equipped with 80Hz
measurement speed and 382 x 288 pixels optical resolution \footnote{http://www.optris.com/thermal-imager-pi400}. With high speed capturing ability, Optris PI450 is sufficient for real-time application, namely Simultaneous Localization and Mapping (SLAM).   

Four datasets with different signal-to-noise (SNR) ratios  are used as follows. 
\begin{itemize}
\item High SNR indoor
\item Low SNR indoor 
\item High SNR outdoor
\item Low SNR outdoor 
\end{itemize}

Thermal images not in our database can be easily added to the training set without affecting any algorithms.








%A significant amount of
%noise is added during the acquisition process (zoom,
%viewpoint, light changes, Jpeg compression). The zoom
%changes involve a change in pixel intensity as automatic
%camera settings are used. Jpeg compression additionally
%introduces artifacts. Some of the image pairs are
%displayed in Section 5.2. In order to use a homography
%for verification we used planar scenes or 3D scenes
%with a fixed camera position. The homography between
%images was estimated using manually selected corresponding
%points. Each scale change sequence consists
%of scaled and rotated images, for which the scale factor
%varies from 1.4 to 4.5. For the viewpoint change
%sequences the viewpoint varies in the horizontal direction
%between 0 and 70 degrees. There are 10 images in
%each sequence representing different scenes. The experiments
%were carried out using 10 scale change sequences
%and 6 viewpoint change sequences of real images,
%one of the sequences is displayed in Fig. 9. There are 
%160 images in total and approximately 100 000 interest
%points are detected in these images and used to
%evaluate the detectors.


%Label and describe each of the datasets captured, with justification (e.g. for the variety).

%Figure \ref{fig:datasets} shows samples of datasets:

%\begin{figure}
%\centering
%	\begin{subfigure}{0.49\columnwidth}
%    \centering
%    \includegraphics[width=1.00\textwidth]{media/dummy.jpg}
%    \caption{Sequence 1}
%		\label{fig:datasets_1}
%  \end{subfigure}
%	\begin{subfigure}{0.49\columnwidth}
%    \centering
%    \includegraphics[width=1.00\textwidth]{media/dummy.jpg}
%		\caption{Sequence 2}
%		\label{fig:datasets_2}
%  \end{subfigure} \vspace{10pt} \\ 
%	\begin{subfigure}{0.49\columnwidth}
%    \centering
%    \includegraphics[width=1.00\textwidth]{media/dummy.jpg}
%    \caption{Sequence 3}
%		\label{fig:datasets_3}
%  \end{subfigure}
%	\begin{subfigure}{0.49\columnwidth}
%    \centering
%    \includegraphics[width=1.00\textwidth]{media/dummy.jpg}
%		\caption{Sequence 4}
%		\label{fig:datasets_4}
%  \end{subfigure} \vspace{10pt} \\ 
%	\begin{subfigure}{0.49\columnwidth}
%    \centering
%    \includegraphics[width=1.00\textwidth]{media/dummy.jpg}
%    \caption{Sequence 5}
%		\label{fig:datasets_5}
%  \end{subfigure}
%	\begin{subfigure}{0.49\columnwidth}
%    \centering
%    \includegraphics[width=1.00\textwidth]{media/dummy.jpg}
%		\caption{Sequence 6}
%		\label{fig:datasets_6}
%  \end{subfigure} \vspace{10pt} \\ 
%	\begin{subfigure}{0.49\columnwidth}
%    \centering
%    \includegraphics[width=1.00\textwidth]{media/dummy.jpg}
%    \caption{Sequence 7}
%		\label{fig:datasets_7}
%  \end{subfigure}
%	\begin{subfigure}{0.49\columnwidth}
%    \centering
%    \includegraphics[width=1.00\textwidth]{media/dummy.jpg}
%		\caption{Sequence 8}
%		\label{fig:datasets_8}
%  \end{subfigure}
%\caption{Datasets for evaluation}
%\label{fig:datasets}
%\end{figure}